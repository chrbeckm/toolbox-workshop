\begin{frame}{Funktionsweise}
  \begin{center}
    \begin{tikzpicture}
      \graph [
        grow down,
        branch right sep,
        nodes={
          node font=\ttfamily,
        },
      ] {
        "all" [x=3.5] -> "report.pdf" [x=3] -> {
          "report.tex",
          "plot.pdf" -> {
            "plot.py", "data.txt"
          },
          "plot2.pdf" -> {
            "plot2.py", "data.txt", "data2.txt"
          },
        },
      };
    \end{tikzpicture}
  \end{center}

  \begin{itemize}
    \item Abhängigkeiten bilden einen DAG (directed acyclic graph/gerichteter azyklischer Graph)
    \item Dateien werden neu erstellt, falls sie nicht existieren oder älter als ihre Prerequisites sind
    \item Prerequisites werden zuerst erstellt
    \item top-down Vorgehen
  \end{itemize}
\end{frame}

\begin{frame}[fragile]{Argumente für \texttt{make}}
  \begin{tblr}{
      colspec = {l X[l]},
      column{1} = {font=\ttfamily},
    }
    make \textit{target} & statt des ersten in der \texttt{Makefile} genannten Targets (meist \texttt{all}) nur \texttt{target} erstellen \\
    make -n              & dry run: Befehle anzeigen aber nicht ausführen \\
    make -B              & Force: ausführen aller Schritte, ignorieren des Alters aller Dateien \\
    make -p              & Datenbank aller Abhängigkeiten ausgeben
  \end{tblr}
  \begin{itemize}
    \item Nützlich, wenn man einen Plot bearbeitet: \texttt{make plot.pdf}
  \end{itemize}
\end{frame}
