\headlineframe{Is the cake a lie?}

\begin{frame}[fragile]{Vergleich: Kuchen backen}
  \begin{center}
    \begin{minted}{make}
    Kuchen: Teig Backofen
        Ofen auf 140°C vorheizen
        Teig in Backform geben und in den Ofen schieben
        Kuchen nach 40 min herausnehmen

    Teig: Eier Mehl Zucker Milch Rumrosinen | Schüssel
        Eier schlagen
        Mehl, Zucker und Milch hinzugeben
        Rumrosinen unterrühren

    Rumrosinen: Rum Rosinen
        Rosinen in Rum einlegen
        Vier Wochen stehen lassen

    Schüssel:
        Rührschüssel auf den Tisch stellen, wenn nicht vorhanden

    clean:
        Kuchen essen
        Küche sauber machen und aufräumen
    \end{minted}
  \end{center}
\end{frame}

\begin{frame}[fragile]{Expert}
  Können mehrere unabhängige Auswertungen parallel ausgeführt werden? \\
  → Ja: \;\texttt{make -j4}\; (nutzt 4 Prozesse (\texttt{j}: jobs) gleichzeitig, Anzahl beliebig)

  Problem: Manchmal führt \texttt{make} Skripte gleichzeitig zweimal aus (hier \texttt{plot.py})
  \begin{center}
    \begin{minted}{make}
      all: report.txt

      report.txt: plot1.pdf plot2.pdf
        touch report.txt

      plot1.pdf plot2.pdf: plot.py data.txt
        python plot.py  # plot.py produziert sowohl plot1.pdf als auch plot2.pdf
    \end{minted}
  \end{center}

  Lösung: manuell synchronisieren
  \begin{center}
    $\vdots$ \\
    \begin{minted}{make}
      plot1.pdf: plot.py data.txt
          python plot.py

      plot2.pdf: plot1.pdf
    \end{minted}
  \end{center}

  Wenn man \texttt{plot2.pdf} aber nicht \texttt{plot1.pdf} löscht, kann \texttt{make} nicht mehr \texttt{plot2.pdf} erstellen.
\end{frame}
