\headlineframe{Projekte \enquote{sauber} halten}

\begin{frame}[fragile]{\texttt{make clean}}
  (Nützliche) Konvention: \texttt{make clean} löscht alle vom \texttt{Makefile} erstellten Dateien/Ordner.

  \begin{center}
    \begin{minted}{make}
      clean:
          rm plot.pdf report.pdf
    \end{minted}
  \end{center}

  Das Projekt sollte dann so aussehen, wie vor dem ersten Ausführen von \texttt{make}.
\end{frame}

\begin{frame}[fragile]{\texttt{build}-Ordner}
  \texttt{build}-Ordner: Projekt sauber halten

  \begin{center}
    \begin{minted}{make}
      all: build/report.pdf

      build/plot.pdf: plot.py data.txt | build
          python plot.py  # savefig('build/plot.pdf')

      build/report.pdf: report.tex build/plot.pdf | build
          lualatex --output-directory=build report.tex

      build:
          mkdir -p build

      clean:
          rm -rf build

      .PHONY: all clean
    \end{minted}
  \end{center}

  \begin{itemize}
    \item \texttt{| build} ist ein order-only Prerequisite: Alter wird ignoriert
    \item Targets, die bei \texttt{.PHONY} genannt werden, erzeugen keine Dateien (guter Stil).
        Bsp: \textbf{\texttt{clean}} löscht Dateien, wird versehentlich eine Datei clean erstellt,
        soll trotzdem \textbf{\texttt{clean}} ausgeführt werden.
        Nennung hier hebt die Verwirrung von \texttt{make} auf, beugt vor.
  \end{itemize}
\end{frame}


\begin{frame}[fragile]{\texttt{build}-Ordner}
  \LuaTeX\ und \texttt{biber} bieten Optionen an, um einen \texttt{build}-Ordner zu benutzen.
  \begin{block}{Aufrufe}
    \begin{minted}{text}
      lualatex --output-directory=build file.tex
      biber build/file.bcf
    \end{minted}
  \end{block}

  Um Dateien aus dem \texttt{build}-Ordner zu finden (Plots, Tabellen):
  \begin{block}{Aufrufe}
    \begin{minted}{text}
      TEXINPUTS=build: lualatex --output-directory=build file.tex
      biber build/file.bcf
    \end{minted}
  \end{block}
  \vspace*{-0.5em}
  \begin{itemize}
    \item \texttt{TEXINPUTS}, \texttt{BIBINPUTS}: Suchpfade für \TeX- und \texttt{.bib}-Dateien
    \item Elemente getrennt mit \texttt{:}, der erste Treffer wird genommen (wie \texttt{PATH})
    \item Hilfreich um z.\,B.\ den Header nur einmal für alle Protokolle abzuspeichern. (Siehe latex-template)
    \item \texttt{TEXINPUTS} auch für \mintinline{latex}+\includegraphics+
    \item \texttt{:} am Ende der Liste: Standardsuchpfade anhängen (wichtig!)
    \item \texttt{.} (der aktuelle Ordner) ist am Anfang der Standardliste, braucht man also nicht selbst angeben
    \item Endet ein Element mit \texttt{//}, werden auch alle Unterordner durchsucht
  \end{itemize}
\end{frame}
