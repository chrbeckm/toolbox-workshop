\begin{frame}{Automatisierte, reproduzierbare Prozesse}

  {\huge Problem:}\\
  \vspace{1em}
  Kurz vor Abgabe noch neue Korrekturen einpflegen
  \begin{enumerate}
    \item Tippfehler korrigieren, Plots bearbeiten
    \item \TeX{} ausführen, ausdrucken
  \end{enumerate}
  \begin{itemize}
    \item vergessen, Plots neu zu erstellen
    \item zurück zu Schritt 1 \dots
  \end{itemize}
\end{frame}

\begin{frame}{Automatisierte, reproduzierbare Prozesse}

  {\huge Lösung: Make}
  \vspace{1em}
  \begin{itemize}
    \item prüft, welche Dateien geändert wurden
    \item berechnet nötige Operationen um Abhängigkeiten zu erfüllen
    \item führt Befehle aus
    \begin{itemize}
      \item Python-Skripte
      \item \TeX{}
      \item etc \dots
    \end{itemize}
  \end{itemize}
\end{frame}

\begin{frame}{Motivation}
  
  {\huge Warum?}
  \vspace{1em}
  \begin{itemize}
    \item \textbf{Automatisierung} verhindert Fehler
    \item Dient als \textbf{Dokumentation}
    \item \textbf{Reproduzierbarkeit}: unverzichtbar in der Wissenschaft
    \item \textbf{Spart Zeit}: nur notwendige Operationen werden ausgeführt
  \end{itemize}
  \vspace{1em}
  \textbf{Ziel:} Eingabe von \texttt{make} erstellt komplettes Protokoll/Paper aus Daten
\end{frame}
