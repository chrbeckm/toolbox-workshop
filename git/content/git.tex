\newcommand{\mail}{\includegraphics[width=0.2\textwidth]{figures/mail.pdf}}
\tikzstyle{arr}=[line width=1mm,draw=black,-triangle 45]

\begin{frame}[c]{Was ist Versionskontrolle?}
  \begin{itemize}
    \item Versionskontrollsoftware speichert Änderungen an Dokumenten / Dateien
    \item Das kann fast alles sein:
      \begin{itemize}
        \item Software
        \item Rechtliche Dokumente
        \item Dokumentation
        \item Wissenschaftliche Veröffentlichungen
        \item Bilder
        \item Baupläne, CAD-Zeichnungen
        \item ...
      \end{itemize}
    \item Ein \emph{Schnappschuss} eines Projektes nennt man \emph{Revision}
    \item Alle Revisionen zusammen bilden die \emph{Geschichte} des Projekts
  \end{itemize}
\end{frame}

\begin{frame}[c]{Warum also Versionskontrolle nutzen?}
  \begin{itemize}
    \item Erlaubt, an eine beliebige Revision zurückzukehren
    \item Kann die Unterschiede zwischen Revisionen anzeigen
    \item Macht Zusammenarbeit an Projekten einfacher
    \item Dient auch als Backup
  \end{itemize}
\end{frame}

\begin{frame}[c]{Warum also Versionskontrolle nutzen?}
  Versionskontrollsoftware macht die Beantwortung der folgenden Fragen einfach:
  \begin{description}[Warum?]
    \item[Was?] Was wurde von Revision \emph{A} auf Revision \emph{B} geändert
    \item[Wer?] Wer hat eine Änderung gemacht? Wer hat alles zum Projekt beigetragen?
    \item[Warum?] Warum wurde diese Änderung gemacht?
    \item[Wann?] Wann wurde ein bestimmter Bug eingeführt bzw. behoben?
  \end{description}

  \onslide<2>{%
    \begin{center}%
      \Large\color{vertexDarkRed}%
      Versionskontrolle ist eine fundamentale Bedingung\\
      für nachvollziehbare, reproduzierbare Wissenschaft.%
    \end{center}
  }
\end{frame}

\headlineframe{Wie arbeitet man am besten an einem Protokoll zusammen?}

\headlineframe{Idee: Austausch über Mails / Messenger}

\begin{frame}{Mails / Messenger: Probleme}
  \centering
  \begin{tikzpicture}
    \node (alice) at (0, 0) {\includegraphics[width=0.2\textwidth]{figures/text.pdf}};
    \node (bob) at (8, 0) {\includegraphics[width=0.2\textwidth]{figures/text.pdf}};
    \draw[arr] (alice) -- node {\mail} (bob);
  \end{tikzpicture}
  \begin{itemize}
    \item Risiko, dass Änderungen vergessen werden, ist groß
    \item Bei jedem Abgleich muss jemand anders aktiv werden
      \begin{itemize}
        \item Stört
        \item Es kommt zu Verzögerungen
      \end{itemize}
  \end{itemize}
  \textbf{\Large Fazit: Eine sehr unbequeme / riskante Lösung}
\end{frame}

\headlineframe{%
  Idee: Austausch über Cloud Speicher\\[\baselineskip]
  Dropbox, Google Drive, OneDrive, Nextcloud, iCloud, ...%
}

\begin{frame}{Cloud Speicher: Probleme}
  \centering
  \begin{tikzpicture}
    \node (alice) at (0, 0) {\includegraphics[width=0.2\textwidth]{figures/text.pdf}};
    \node[align=center,fill=blue!20,rounded corners=1em] (dropbox) at (4, 0.5) {
      \includegraphics[width=0.1\textwidth]{logos/dropbox.pdf}\\
      \includegraphics[width=0.2\textwidth]{figures/text.pdf}
    };
    \node (bob) at (8, 0) {\includegraphics[width=0.2\textwidth]{figures/text.pdf}};
    \draw[arr,triangle 45-triangle 45] (0.8, 0) -- (dropbox);
    \draw[arr,triangle 45-triangle 45] (dropbox) -- (7.2, 0);
  \end{tikzpicture}

  \begin{itemize}
    \item Man merkt nichts von Änderungen der Anderen
    \item Gleichzeitige Änderungen führen zu \enquote{In Konflikt stehende Kopie}-Dateien
    \item Änderungen werden nicht zusammengeführt
    \item Keine echte Historie des Projekts
  \end{itemize}
  \textbf{\Large Fazit: Besser, aber hat deutliche Probleme}
\end{frame}

\headlineframe{Lösung: Änderungen verwalten mit \texttt{git}}

\begin{frame}
    \centering
    \includegraphics[width=0.7\textwidth]{logos/git_logo.pdf}

    \vspace{1em}

    \begin{itemize}
      \item Ein Versionskontrollsystem
      \item Ursprünglich entwickelt, um den Programmcode des Linux-Kernels zu verwalten (Linus Torvalds)
      \item Hat sich gegenüber ähnlichen Programmen (SVN, mercurial) durchgesetzt
      \item Wird in der Regel über die Kommandozeile benutzt
      \item Es gibt auch Plugins für Editoren, z.B. VS Code
    \end{itemize}
\end{frame}

\begin{frame}{Was bringt \texttt{git} für Vorteile?}
  \begin{itemize}
    \item Arbeit wird für andere sichtbar protokolliert
    \item Erlaubt Zurückspringen an einen früheren Zeitpunkt
    \item Kann die meisten Änderungen automatisch zusammenfügen
    \item Wirkt nebenbei auch als Backup
  \end{itemize}
  Einzige Herausforderung: Man muss lernen, damit umzugehen
\end{frame}

\begin{frame}{Zentrales Konzept: Das Repository}
  \begin{itemize}
    \item Erzeugen mit \texttt{git init}
    \item Damit wird der aktuelle Ordner zu einem Repository
  \end{itemize}
  \vspace{3em}
  \centering
  \begin{tikzpicture}[
      line width=1.5,
      gitstep/.style={
        draw,
        rounded corners,
        thick,
        minimum width=4cm,
        minimum height=1.2cm,
      },
    ]
    \node (wd) at (0, 0) [gitstep, fill=red!20, visible on=<1->, align=center] {Working directory \\{\includegraphics[width=.03\textwidth]{figures/paper.png}}};
    \node (idx) [gitstep, fill=yellow!20, below=0.4cm of wd, visible on=<2->, align=center] {Staging \\{\includegraphics[width=.03\textwidth]{figures/baggage.png}}};
    \node (hist) [gitstep, fill=green!20, below=0.4cm of idx, visible on=<3->, align=center] {History \\{\includegraphics[width=.03\textwidth]{figures/baggage.png}\hspace{0.1cm}\includegraphics[width=.03\textwidth]{figures/baggage.png}\hspace{0.1cm}\includegraphics[width=.03\textwidth]{figures/baggage.png}}};
    \node [right=1cm of wd, text width=0.5\textwidth, align=flush left, visible on=<1->] {Aktuelles Arbeitsverzeichnis, Inhalt des Ordners im Dateisystem.};
    \node [right=1cm of idx, text width=0.5\textwidth, align=flush left, visible on=<2->] {Änderungen, die für einen \enquote{commit} vorgemerkt sind.};
    \node [right=1cm of hist, text width=0.5\textwidth, align=flush left, visible on=<3->] {Gespeicherte \emph{Historie} des Projekts. Alle jemals gemachten Änderungen. Ein Baum von Commits.};
    \draw[thick,->, visible on=<4->] (wd.west) to[out=180,in=180] node[left] {\texttt{git add}} (idx.west);
    \draw[thick,->, visible on=<5->] (idx.west) to[out=180,in=180] node[left] {\texttt{git commit}} (hist.west);
  \end{tikzpicture}
\end{frame}

\begin{frame}{Remotes}
  Remotes sind zentrale Stellen, z.\,B. Server auf denen die History gespeichert wird.
  \begin{center}
  \begin{tikzpicture}[line width=1.5]
    \node (hist) [draw,rounded corners,thick,minimum width=4cm,minimum height=1.2cm,fill=green!20, align=center] {Eure History \\{\includegraphics[width=.03\textwidth]{figures/baggage.png}\hspace{0.1cm}\includegraphics[width=.03\textwidth]{figures/baggage.png}\hspace{0.1cm}\includegraphics[width=.03\textwidth]{figures/baggage.png}}};
    \node (remote) [below=0.4cm of hist,draw,rounded corners,thick,minimum width=4cm,minimum height=1.2cm,fill=red!20, align=center] {Remote \\{\includegraphics[width=.03\textwidth]{figures/baggage.png}\hspace{0.1cm}\includegraphics[width=.03\textwidth]{figures/baggage.png}\hspace{0.1cm}\includegraphics[width=.03\textwidth]{figures/baggage.png}\hspace{0.03\textwidth}\includegraphics[width=.03\textwidth]{figures/baggage.png}}};
    \node (hist2) [visible on=<2->,below=0.4cm of remote,draw,rounded corners,thick,minimum width=4cm,minimum height=1.2cm,fill=blue!20, align=center] {History eures Partners \\{\includegraphics[width=.03\textwidth]{figures/baggage.png}\hspace{0.1cm}\includegraphics[width=.03\textwidth]{figures/baggage.png}\hspace{0.1cm}\includegraphics[width=.03\textwidth]{figures/baggage.png}}};
    \draw[thick,->] (hist.east) to[out=0,in=0] node[right] {\texttt{git push}} (remote.east);
    \draw[thick,->] (remote.west) to[out=180,in=180] node[left] {\texttt{git pull}} (hist.west);
  \end{tikzpicture}
  \end{center}
\end{frame}

\begin{frame}{Konzept: Commits}
  \begin{itemize}
    \item Stand des Repositories zu einem Zeitpunkt
    \item Erlaubt das Hinzufügen von Kommentaren: Was wurde getan seit dem letzten Commit?
    \item Sind die \textit{Versionen} des Repositories
  \end{itemize}
  \begin{tikzpicture}
    \node (commit1) at (0, 0) {{\includegraphics[width=0.2\textwidth]{figures/baggage.png}}};
    \node (commit2) at (8, 0) {{\includegraphics[width=0.2\textwidth]{figures/baggage.png}}};
    \node (commit1_text) [below=0.5cm of commit1] {\texttt{Commit 1}};
    \node (commit2_text) [below=0.5cm of commit2] {\texttt{Commit 2}};
    \draw[arr] (commit1) -- (commit2);
  \end{tikzpicture}
\end{frame}

\begin{frame}{History}
  \only<1>{
    \begin{tikzpicture}
      \graph [
        grow right=1.5cm,
        node distance=0.5cm,
        nodes={
          blue!70!black,
          node font=\ttfamily,
        },
      ]{
        "Erstabgabe" [visible on=<4->, green!60!black, x=4];
		a <- b <- c <- d <- main [vertexDarkRed];
        f[visible on=<2>] ->[visble on=<2>] -> a;
        "Erstabgabe"[visible on=<2>] ->[visible on=<3>] c;
      };
    \end{tikzpicture}
  }
  \only<2>{
    \begin{tikzpicture}
      \graph [
        grow right=1.5cm,
        node distance=0.5cm,
        nodes={
          blue!70!black,
          node font=\ttfamily,
        },
      ]{
        "Erstabgabe" [visible on=<4->, green!60!black, x=4];
        a <- b <- c <- {
          d <- main [vertexDarkRed],
          e <- "fix\_plot" [vertexDarkRed]
        };
        "Erstabgabe"[visible on=<3>] ->[visible on=<3>] c;
      };
    \end{tikzpicture}
  }
  \only<3->{
    \begin{tikzpicture}
      \graph [
        grow right=1.5cm,
        nodes={
          blue!70!black,
          node font=\ttfamily,
        },
      ] {
        "Erstabgabe" [visible on=<4->, green!60!black, x=6],
        a <- b <- c <- {
          d <- f,
          e <- g [x=0.5],
        } <- h <- {i <-[visible on=<2->] main [vertexDarkRed, visible on=<3->], j <-[visible on=<2->] "fix\_formula" [vertexDarkRed, visible on=<3->]};
        "Erstabgabe" ->[visible on=<4->] f;
      };
    \end{tikzpicture}
  }

  \vspace{1em}
  \begin{itemize}
    \item<1-> \textcolor{blue}{Commit}: Zustand/Inhalt des Arbeitsverzeichnisses zu einem Zeitpunkt
      \begin{itemize}
        \item Enthält Commit-Message (Beschreibung der Änderungen)
        \item Wird über einen Hash-Code identifiziert
        \item Zeigt immer auf seine(n) Vorgänger
      \end{itemize}
    \item<2-> \textcolor{vertexDarkRed}{Branch}: benannter Zeiger auf einen Commit
      \begin{itemize}
        \item Entwicklungszweig
		\item Im Praktikum reicht meist der Standard-Branch: \texttt{main}
		    \item Wandert weiter mit dem aktuellsten Commit
      \end{itemize}
    \item<4-> \textcolor{green!60!black}{Tag}: unveränderbarer Zeiger auf einen Commit
      \begin{itemize}
        \item Wichtiges Ereignis, z.B. veröffentlichte Version
      \end{itemize}
  \end{itemize}
\end{frame}

\begin{frame}{Typischer Arbeitsablauf}
  \begin{enumerate}
    \item Neues Repo? Repository erzeugen oder klonen: \hfill\texttt{git init}, \texttt{git clone} \\
      Repo schon da? Änderungen herunterladen: \hfill\texttt{git pull}
    \item Arbeiten
      \begin{enumerate}
        \item Dateien bearbeiten und testen
        \item Änderungen vorbereiten: \hfill\texttt{git add}
        \item Änderungen als \emph{commit} speichern: \hfill\texttt{git commit}
      \end{enumerate}
    \item Commits anderer herunterladen und integrieren: \hfill\texttt{git pull}
    \item Eigene Commits hochladen: \hfill\texttt{git push}
  \end{enumerate}
\end{frame}

\begin{frame}{Zum selber ausprobieren und Lernen:}
  \begin{figure}
    \centering
    \includegraphics[width=.6\textwidth]{figures/learngitbranching.png}
  \end{figure}
  \texttt{https://learngitbranching.js.org/}
\end{frame}

\begin{frame}{Der Start: \texttt{git init}, \texttt{git clone}}
  \begin{tblr}{
      colspec = {l X[l]},
      column{1} = {font=\ttfamily},
      row{3} = {font=\color{lightgray}},
    }
    git init               & initialisiert ein \texttt{git}-Repo im jetzigen Verzeichnis \\
    git clone \textit{url} & klont das Repo aus \texttt{\textit{url}} \\
    rm -rf .git            & löscht alle Spuren von \texttt{git} aus dem Repository, nicht reversibel ohne Backup, wird eigentlich nie gebraucht
  \end{tblr}
\end{frame}

\begin{frame}{Was passiert in Git: \texttt{git status}, \texttt{git log}}
  \begin{tblr}{
      colspec = {l X[l]},
      column{1} = {font=\ttfamily}, 
    }
    git status    & zeigt Status des Repos (welche Dateien sind neu, gelöscht, verschoben, bearbeitet) \\
    git status -s & Kurzform von \texttt{git status}, zeigt Liste von geänderten Dateien \\
    git log      & listet Commits in aktuellem Branch. Hat viele Optionen.
  \end{tblr}
\end{frame}

\begin{frame}{Staging Bereich: \texttt{git add}, \texttt{git mv}, \texttt{git rm}, \texttt{git reset}}
  \begin{tblr}{
      colspec = {l X[l]},
      column{1} = {font=\ttfamily}, 
    }
    git add \textit{file} … & fügt Dateien/Verzeichnisse zum Staging-Bereich hinzu \\
    git add -p …            & fügt Teile einer Datei zum Staging-Bereich hinzu \\
    git add -u …            & fügt \emph{alle} von Git getrackten und vom User veränderten Dateien zum Staging-Bereich hinzu\\
    git mv                  & wie \texttt{mv} (automatisch in Staging)\\
    git rm                  & wie \texttt{rm} (automatisch in Staging) \\
    git reset \textit{file} & entfernt Dateien/Verzeichnisse aus Staging
  \end{tblr}
\end{frame}

\begin{frame}{\texttt{git diff}}
  \begin{tblr}{
      colspec = {l X[l]},
      column{1} = {font=\ttfamily}, 
    }
    git diff                                   & zeigt Unterschiede zwischen Staging und Arbeitsverzeichnis \\
    git diff --staged                          & zeigt Unterschiede zwischen letzten Commit und Staging \\
    git diff \textit{commit1} \textit{commit2} & zeigt Unterschiede zwischen zwei Commits
  \end{tblr}
\end{frame}

\begin{frame}{\texttt{git commit}}
  \begin{tblr}{
      colspec = {l X[l]},
      column{1} = {font=\ttfamily}, 
    }
    git commit                       & erzeugt Commit aus jetzigem Staging-Bereich, öffnet Editor für Commit-Message \\
    git commit -m "\textit{message}" & Commit mit \texttt{\textit{message}} als Message \\
    git commit --amend               & letzten Commit ändern (fügt aktuellen Staging hinzu, Message bearbeitbar) \\
    & \alert{\bfseries Niemals commits ändern, die schon in den \texttt{main} branch gepusht sind!}
  \end{tblr}

  \begin{itemize}
    \item Wichtig: Sinnvolle Commit-Messages
      \begin{itemize}
        \item Erster Satz ist Zusammenfassung (ideal < 50 Zeichen)
        \item Danach eine leere Zeile lassen
        \item Dann längere Erläuterung des commits
      \end{itemize}
    \item Logische Commits erstellen, für jede logische Einheit ein Commit
      \begin{itemize}
        \item \texttt{git add -p} ist hier nützlich
      \end{itemize}
    \item Hochgeladene Commits sollte man nicht mehr ändern
  \end{itemize}
\end{frame}

\begin{frame}{Mit der remote History (dem Server) interagieren}
  \begin{tblr}{
      colspec = {l X[l]},
      column{1} = {font=\ttfamily}, 
    }
    git pull          & Commits herunterladen \\
    git push          & Commits hochladen
  \end{tblr}
\end{frame}

\begin{frame}[fragile]{Achtung: Merge conflicts}
  \begin{center}
    \huge Don't Panic
  \end{center}

  Entstehen, wenn \texttt{git} nicht automatisch mergen kann (selbe Zeile geändert, etc.)

  \begin{enumerate}
    \item Die betroffenen Dateien öffnen
    \item Markierungen finden und die Stelle selbst mergen (meist wenige Zeilen)
      \begin{verbatim}
        <<<<<<< HEAD
        foo
        ||||||| merged common ancestors
        bar
        =======
        baz
        >>>>>>> Commit-Message
\end{verbatim}
    \item Merge abschließen:
      \begin{enumerate}
          \item \texttt{git add …} \quad (Files mit behobenen Konflikten)
        \item \texttt{git commit} \quad\rightarrow\; Editor wird geöffnet
        \item Vorgeschlagene Nachricht kann angenommen werden (In vim \texttt{":wq"} eintippen)
      \end{enumerate}
  \end{enumerate}
  Nützlich: \texttt{git config --global merge.conflictstyle diff3}
\end{frame}

\begin{frame}{Zu früheren Versionen zurückkehren}
  \begin{tblr}{
      colspec = {l X[l]},
      column{1} = {font=\ttfamily}, 
    }
    git checkout \textit{commit} & Commit ins Arbeitsverzeichnis laden \\
    git restore \textit{filename} & Änderungen an Dateien verwerfen (zum letzten Commit zurückkehren)
  \end{tblr}
\end{frame}

\begin{frame}{Kurz an was anderem Arbeiten}
  \begin{tblr}{
      colspec = {l X[l]},
      column{1} = {font=\ttfamily}, 
    }
    git stash     & Änderungen kurz zur Seite schieben \\
    git stash pop & Änderungen zurückholen aus Stash
  \end{tblr}
\end{frame}

\begin{frame}[fragile]{\texttt{.gitignore}}
    \begin{itemize}
    \item Man möchte nicht alle Dateien von \texttt{git} beobachten lassen
    \item z.B. \texttt{build}-Ordner
    \end{itemize}
    \begin{center}
        \Large Lösung: \texttt{.gitignore}-Datei
    \end{center}

    \begin{itemize}
    \item einfache Textdatei
    \item enthält Regeln für Dateien, die nicht beobachtet werden sollen
    \end{itemize}
    Beispiel:
    \vspace{1em}
    \begin{minted}{text}
      build/
      *.pdf
      __pycache__/
    \end{minted}
\end{frame}


\begin{frame}{Hoster}
  \begin{tblr}{
      colspec = {X[c] X[c] X[c]},
      measure=vbox,
    }
    \href{https://github.com}{\includegraphics[width=0.75\linewidth]{figures/github.png}} &
    \href{https://bitbucket.org}{\includegraphics[width=0.75\linewidth]{figures/bitbucket.png}} &
    \href{https://gitlab.com}{\includegraphics[width=0.75\linewidth]{figures/gitlab.png}} \\
    \begin{itemize}
      \item größter Hoster
      \item viele open-source Projekte
      \item Unbegrenzt private Repositories für Studenten und Forscher:  \newline
        \href{http://education.github.com}{education.github.com}
    \end{itemize}
    &
    \begin{itemize}
      \item kostenlose private Repos mit höchstens fünf Leuten
      \item keine Speicherbegrenzungen
      \item Hängt was Oberfläche und Funktionen angeht, den beiden anderen weit hinterher
    \end{itemize}
    &
    \begin{itemize}
      \item open-source
      \item keine Begrenzungen an privaten Repos
      \item kann man selbst auf einem eigenen Server betreiben
    \end{itemize}\\
  \end{tblr}

  Weitere Open Source Optionen, auch zum selbst hosten: Gitea, Forgejo

  \begin{center}
    \onslide<2->{%
      \Large%
      \enquote{Now, everybody sort of gets born with a GitHub account}
      – Guido van Rossum
    }
  \end{center}
\end{frame}

\begin{frame}[fragile]{SSH-Keys}
  Git kann auf mehrere Arten mit einem Server kommunizieren:
  \begin{description}
    \item[HTTPS]
      \begin{itemize}
        \item Mit Nutzername / Passwort:
          War lange die einfachste Möglichkeit. Wird aber von GitHub aus Sicherheitsgründen nicht mehr einfach unterstützt.
        \item Mit \enquote{Personal Access Token}. Neues Verfahren für GitHub über HTTPS.
      \end{itemize}
    \item[SSH]: Keys müssen erzeugt und eingestellt werden, Passwort für den Key muss, wenn ein \enquote{SSH-Agent} verwendet wird, nur einmal pro Session eingegeben werden.
  \end{description}

  SSH-Keys:
  \begin{enumerate}
    \item \mintinline{text}+ssh-keygen -t ed25519 -C "your_email@example.com"+
    \item Passwort wählen
    \item \mintinline{text}+cat ~/.ssh/id_ed25519.pub+
    \item Ausgabe ist Public-Key, beim Server eintragen (im Browser)
  \end{enumerate}

  Für den Agent, falls noch nicht vom Betriebsystem eingerichtet (z.\,B.\ Windows mit WSL):
  \begin{enumerate}
    \setcounter{enumi}{4}
    \item \mintinline{text}+echo 'eval $(ssh-agent -s)' >> ~/.bashrc+
    \item \mintinline{text}+echo 'AddKeysToAgent yes' >> ~/.ssh/config+
  \end{enumerate}

  Doku: \url{https://docs.github.com/en/authentication/connecting-to-github-with-ssh}
\end{frame}

\begin{frame}{Extra Slide für sauberere Projekt Historien: \texttt{git pull --rebase} (optional)}
    Vielfaches Merging und Merge Konflikte erzeugen eine etwas nichtlineare Projekt-Historie, denn:\\
    \texttt{git pull} entspricht \texttt{git fetch origin; git merge …} (\to\;gemergter Branch bleibt erhalten)

    Alternativ kann man \texttt{\textbf{git pull --rebase}} ausführen, welches (in etwa) äquivalent ist zu\\\texttt{git fetch origin; git rebase …} (\to\;lokale Commits werden auf neue Commits angewendet).

    {\color{red} Achtung: Um einen Merge Konflikt bei \texttt{git pull --rebase} abzuschließen, muss \texttt{\textbf{git~rebase~--continue}} \emph{anstelle} von \texttt{git commit -m "..."} ausgeführt werden! Also einfach genau lesen was Git empfiehlt ;)}

    Dies hat Vorteile:
    \begin{itemize}
        \item Die Projekt-Historie ist linearer
        \item Es gibt weniger merge-commits
    \end{itemize}
    aber auch (kleinere) Nachteile:
    \begin{itemize}
        \item Es ist hinterher nicht mehr sichtbar, wer einen Merge Konflikt wie behoben hat
        \item Die Abfolge der Commits entspricht nicht mehr der wahren Entwicklungshistorie
    \end{itemize}
    Entscheidet man sich für pulls mit Rebase als Standard, muss Git anders konfiguriert werden:\\
    \texttt{\textbf{git config --global pull.rebase true}}, dann wird bei allen folgenden \texttt{git pull} Befehlen ein Rebase gemacht
\end{frame}

\begin{frame}{Video-Aufzeichnung}
  Im Rahmen einer Schulung ist 2021 eine Videoaufzeichnung einer ausführlicheren Git-Einführung angefertigt worden, die auf diesem Kurs basiert:

  \begin{description}
    \item[Teil 1] \url{https://www.youtube.com/watch?v=R2BCOtPwtXc}
    \item[Teil 2] \url{https://www.youtube.com/watch?v=ZEcklfIp6Og}
  \end{description}
\end{frame}
