\section{Ablauf}

\begin{frame}{Roadmap}
    \begin{center}
      \begin{tikzpicture}[
          >=latex,
          arrow/.style={line width=2pt, ->}
        ]
        %% Data table
        \node (data) at (0, 0) {
          \sffamily
          \sisetup{table-format=1.2, detect-family}
          \begin{tblr}{
              colspec = {S S[table-format=2.2]},
              row{1} = {guard, mode=math},
              row{7} = {guard, mode=math}
            }
            \toprule
            \mathsf{t} \mathbin{/} \unit{\second} & \mathsf{U} \mathbin{/} \unit{\milli\volt} \\
            \midrule
            0.00 & -68.00 \\
            0.01 & -67.67 \\
            0.02 & -67.34 \\
            0.03 & -67.01 \\
            0.04 & -66.69 \\
            \vdots & \vdots \\
            \bottomrule
          \end{tblr}
        };
        \node [anchor=west, align=center] (plot) at (3, 0) {
          \includegraphics[width=5cm]{build/example_plot.pdf}\\
          \(I_0 = \qty{26.9 \pm 0.5}{\micro\ampere}\)
        };

        %% Protocol
        \node [anchor=west] (protocol) at (10, 0) {
          \begin{tikzpicture}[
              scale=2,
              protocol text/.style={line width=2pt, line cap=round, color=gray}
            ]

            \draw (0,0) -- ++(1.41, 0) -- ++(0, 2)
              -- ++(-1.21, 0) -- (0, 1.8) -- cycle;
            \draw (0, 1.8) -- ++(0.2,0) -- ++(0, 0.2);
            \draw [protocol text] (0.1, 1.7) -- +(1, 0);
            \draw [protocol text] (1.2, 1.7) -- +(0.1, 0);
            \draw [protocol text] (0.1, 1.6) -- +(0.3, 0);
            \draw [protocol text] (0.5, 1.6) -- +(0.5, 0);
            \draw [protocol text] (1.1, 1.6) -- +(0.2, 0);
            \draw [protocol text] (0.1, 1.5) -- +(0.2, 0);
            \draw [protocol text] (0.4, 1.5) -- +(0.9, 0);
            \draw [protocol text] (0.1, 1.4) -- +(0.9, 0);
            \draw [protocol text] (1.1, 1.4) -- +(0.2, 0);
            \draw [protocol text] (0.1, 1.3) -- +(0.5, 0);
            \draw [protocol text] (0.7, 1.3) -- +(0.6, 0);
            \node [anchor=north west] at (0.19, 1.2) {
              \includegraphics[width=1.7cm]{build/example_plot.pdf}
            };
            \draw [protocol text] (0.1, 0.4) -- +(0.6, 0);
            \draw [protocol text] (0.8, 0.4) -- +(0.2, 0);
            \draw [protocol text] (1.1, 0.4) -- +(0.2, 0);
            \draw [protocol text] (0.1, 0.3) -- +(0.3, 0);
            \draw [protocol text] (0.5, 0.3) -- +(0.5, 0);
            \draw [protocol text] (1.1, 0.3) -- +(0.2, 0);
            \draw [protocol text] (0.1, 0.2) -- +(0.7, 0);
            \draw [protocol text] (0.9, 0.2) -- +(0.4, 0);
          \end{tikzpicture}
        };


        %% Arrows
        \draw [arrow] (data)
          -- node[above, midway] {\includegraphics[width=1.5cm]{../common/logos/python.pdf}}
          (plot);
        \draw [arrow] (plot) -- node[above, midway] {\LaTeX} (protocol);

        %% Annotations
        \node at (0, 2.5) {Data};
        \node at (6, 2.5) {Plots und Ergebnisse};
        \node at (11.5, 2.5) {Protokoll (PDF)};
      \end{tikzpicture}
    \end{center}
\end{frame}

\begin{frame}{Ablauf}
  \begin{description}[Nächste Woche]
    \item[Montag] Programmieren mit Python
    \item[Dienstag] Datenhandhabung / Erstellen von Plots
      \begin{itemize}
        \item NumPy
        \item matplotlib
      \end{itemize}
    \item[Mittwoch] Datenauswertung / Fehlerrechnung
      \begin{itemize}
        \item scipy
        \item uncertainties
      \end{itemize}
    \item[Donnerstag] Kommandozeile und Versionskontrolle
      \begin{itemize}
        \item Unix
        \item git
      \end{itemize}
    \item[Freitag] Einstieg in \LaTeX
    \item[Nächste Woche] Verfassen wissenschaftlicher Texte mit \LaTeX{}
      \begin{itemize}
        \item Fließtext \& Mathematik
        \item Referenzen \& Literaturverzeichnis
      \end{itemize}
      \item Automatisierung mit make
      \item Kombination aller gezeigten Tools
      \item Protokollvorlage und abschließende Übungen
  \end{description}
\end{frame}
