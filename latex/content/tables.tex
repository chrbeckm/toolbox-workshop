\section{Tabellen}

\begin{frame}[fragile]{
  Tabellen
  \hfill
  \doc{http://mirrors.ctan.org/macros/latex/contrib/tabularray/tabularray.pdf}{tabularray}
}
  \begin{columns}[t, onlytextwidth]
    \begin{column}{0.33\textwidth}
      \begin{Packages}[equal height group=tabs]
        \begin{minted}{latex}
          \usepackage{tabularray}
          \UseTblrLibrary{booktabs}
        \end{minted}
      \end{Packages}
    \end{column}
    \begin{column}{0.66\textwidth}
    \begin{block}{Neue Klassenoption}[equal height group=tabs]
      \begin{minted}{latex}
        \documentclass[…, captions=tableheading]{scrartcl}
      \end{minted}
    \end{block}
    \end{column}
  \end{columns}
  \vspace{-2pt}
  \begin{columns}[onlytextwidth, t]
    \begin{column}{0.60\textwidth}
      \fontsize{8}{6}
      \begin{block}{Code}
        \begin{minted}{latex}
          \begin{table}
            \centering
            \caption{Eine Tabelle mit Messdaten.}
            \label{tab:some_data}
            \begin{tblr}{colspec={c c c c c}}
              \toprule
              $f$ & $l_\text{start}$ & $l_1$ & $l_{\text{kor},1}$ & $B_1$ \\
              \midrule
              100 & 1.14 & 3.51 & 0.00 &   4.30 \\
              300 & 1.27 & 2.42 & 0.13 &  41.14 \\
              500 & 1.21 & 1.70 & 0.25 & 168.73 \\
              \bottomrule
            \end{tblr}
          \end{table}
        \end{minted}
      \end{block}
    \end{column}
    \begin{column}{0.36\textwidth}
      \begin{itemize}
        \item Äußere \texttt{table}-Umgebung behandelt Tabelle wie ein float
        \item Innere \texttt{tblr}-Umgebung für eigentlichen Tabelleninhalt
        \item \texttt{colspec} mit Keys \texttt{l}, \texttt{c} oder \texttt{r} gibt Ausrichtungen der einzelnen Spalten an
        \item \mintinline{latex}+\caption+, \mintinline{latex}+\label+ oberhalb von \texttt{tblr}
      \end{itemize}
    \end{column}
  \end{columns}
\end{frame}

\begin{frame}{Ergebnis}
  \begin{EmulateArticle}
    \begin{table}
      \centering
      \caption{Eine Tabelle mit Messdaten.}
      \begin{tblr}{colspec={c c c c c}}
        \toprule
        $f$ & $l_\text{start}$ & $l_1$ & $l_{\text{kor},1}$ & $B_1$ \\
        \midrule
        100 & 1.14 & 3.51 & 0.00 &   4.30 \\
        300 & 1.27 & 2.42 & 0.13 &  41.14 \\
        500 & 1.21 & 1.70 & 0.25 & 168.73 \\
        \toprule
      \end{tblr}
    \end{table}
  \end{EmulateArticle}
  \begin{itemize}
    \item Keine vertikalen Linien!
    \item Keine horizontalen Linien zwischen Daten!
  \end{itemize}
\end{frame}

\begin{frame}[fragile]{
  Schönere Tabellen mit \texttt{siunitx}
  \hfill
  \doc{http://mirrors.ctan.org/macros/latex/contrib/siunitx/siunitx.pdf}{siunitx}
}
  \fontsize{8}{6}
  \begin{block}{Benötigte Paketoption}
    \begin{minted}{latex}
       \UseTblrLibrary{siunitx} % Lädt siunitx und definiert die S-Spalte
     \end{minted}
  \end{block}

  \begin{block}{Code}
    \begin{minted}{latex}
      \begin{table}
        \centering
        \caption{Eine schöne Tabelle mit Messdaten.}
        \label{tab:some_data}
        \sisetup{table-format=1.2}
        \begin{tblr}{
            colspec = {S[table-format=3.0] S S S S[table-format=3.2]},
            row{1} = {guard, mode=math},
          }
          \toprule
          f & l_\text{start} & l_1 & l_{\text{kor},1} & B_1 \\
          \midrule
          100 & 1.14 & 3.51 & 0.00 &   4.30 \\
          200 & 1.30 & 2.99 & 0.06 &  25.98 \\
          300 & 1.27 & 2.42 & 0.13 &  41.14 \\
          400 & 1.28 & 1.47 & 0.20 &  53.76 \\
          500 & 1.21 & 1.70 & 0.25 & 168.73 \\
          \bottomrule
        \end{tblr}
      \end{table}
    \end{minted}
  \end{block}
\end{frame}

\begin{frame}[fragile]{Ergebnis}
  \begin{EmulateArticle}
    \begin{table}
      \centering
      \caption{Eine schöne Tabelle mit Messdaten.}
      \sisetup{table-format=1.2}
      \begin{tblr}{
          colspec = {S[table-format=3.0] S S S S[table-format=3.2]},
          row{1} = {guard, mode=math},
        }
        \toprule
        f & l_\text{start} & l_1 & l_{\text{kor},1} & B_1 \\
        \midrule
        100 & 1.14 & 3.51 & 0.00 &   4.30 \\
        200 & 1.30 & 2.99 & 0.06 &  25.98 \\
        300 & 1.27 & 2.42 & 0.13 &  41.14 \\
        400 & 1.28 & 1.47 & 0.20 &  53.76 \\
        500 & 1.21 & 1.70 & 0.25 & 168.73 \\
        \bottomrule
      \end{tblr}
    \end{table}
  \end{EmulateArticle}
  \begin{itemize}
    \item \texttt{S}-Spalte eröffnet mehr Ausrichtungsmöglichkeiten mit \mintinline{latex}+\sisetup+ und \mintinline{latex}+[...]+
    \item Standard: Ausrichtung an Dezimalkomma
    \item \mintinline{latex}+{mode=math}+ setzt alle Einträge einer Zeile (oder Spalte) in eine Matheumgebung (Kurzform: \texttt{\$})
    \item Spaltennamen durch \mintinline{latex}+row{Zeile(n)} = {guard}+ schützen (z.\,B. \mintinline{latex}+row{1} = {guard}+)
  \end{itemize}
\end{frame}

\begin{frame}[fragile]{Gruppieren von mehreren Spalten}
  \begin{block}{Kommandostruktur}
    \begin{minted}{latex}
      \SetCell[c=#Spalten]{Ausrichtung} Inhalt
    \end{minted}
  \end{block}
  \fontsize{8}{6}
  \begin{block}{Beispiel}
    \begin{minted}{latex}
      \begin{table}
        \centering
        \caption{Messdaten für dubiose Elemente.}
        \sisetup{table-format=2.1}
        \begin{tblr}{
            colspec = {S[table-format=3.1] S S S S},
            row{1} = {guard}, row{2} = {guard, mode=math},
          }
          \toprule
          & \SetCell[c=2]{c} Technetium & & \SetCell[c=2]{c} Molybdän & \\
          \cmidrule[lr]{2-3}\cmidrule[lr]{4-5}
          \lambda \mathbin{/} \unit{\nano\meter} & \phi_1 & \phi_2 & \phi_1 & \phi_2 \\
          \midrule
          663.0 & 12.1 & 14.4 & 13.1 & 16.9 \\
          670.0 & 10.9 & 12.9 & 11.8 & 15.7 \\
          678.0 &  9.1 & 11.4 & 10.3 & 14.6 \\
          684.0 &  8.2 & 10.2 &  9.5 & 13.5 \\
          \bottomrule
        \end{tblr}
      \end{table}
    \end{minted}
  \end{block}
\end{frame}

\begin{frame}{Resultat}
  \begin{EmulateArticle}
    \begin{table}
      \centering
      \caption{Messdaten für dubiose Elemente.}
      \sisetup{table-format=2.1}
      \begin{tblr}{
          colspec={S[table-format=3.1] S S S S},
          row{1} = {guard}, row{2} = {guard, mode=math},
        }
        \toprule
        & \SetCell[c=2]{c} Technetium & & \SetCell[c=2]{c} Molybdän & \\
        \cmidrule[lr]{2-3}\cmidrule[lr]{4-5}
        \lambda \mathbin{/} \unit{\nano\meter} & \phi_1 & \phi_2 & \phi_1 & \phi_2 \\
        \midrule
        663.0 & 12.1 & 14.4 & 13.1 & 16.9 \\
        670.0 & 10.9 & 12.9 & 11.8 & 15.7 \\
        678.0 &  9.1 & 11.4 & 10.3 & 14.6 \\
        684.0 &  8.2 & 10.2 &  9.5 & 13.5 \\
        \bottomrule
      \end{tblr}
    \end{table}
  \end{EmulateArticle}

  \begin{itemize}
    \item Einheiten werden im Tabellenkopf herausdividiert.
    \item \alert{Achtung: Bei normalem Text im Spaltennamen keine \mintinline{latex}+{{mode=math}}+-Option, da dies den Text ebenfalls in die Matheumgebung setzt.}
  \end{itemize}
\end{frame}

\begin{frame}[fragile]{Unsicherheiten in Tabellen}
  \begin{block}{Neue Option}
    \begin{minted}{latex}
      vline{Spalte(n)} = {Zeile(n)}{text=Symbol}
    \end{minted}
  \end{block}
  \begin{block}{Beispiel}
    \begin{minted}{latex}
      \begin{tblr}{
          colspec = {S[table-format=3.1] S[table-format=2.1]},
          row{1} = {guard, mode=math},
          vline{2} = {2}{-}{text=\clap{$\pm$}},
        }
        \toprule
        \SetCell[c=2]{c} x \mathbin{/} \unit{\ohm} \\
        \midrule
        663.0 & 12.1 \\
        670.0 & 10.9 \\
        678.0 &  9.1 \\
        684.0 &  8.2 \\
        \bottomrule
      \end{tblr}
    \end{minted}
  \end{block}
\end{frame}

\begin{frame}{Resultat}
  \begin{center}
    \begin{tblr}{
        colspec = {S[table-format=3.1] S[table-format=2.1]},
        row{1} = {guard, mode=math},
        vline{2} = {2}{-}{text=\clap{$\pm$}},
      }
      \toprule
      \SetCell[c=2]{c} x \mathbin{/} \unit{\ohm} \\
      \midrule
      663.0 & 12.1 \\
      670.0 & 10.9 \\
      678.0 &  9.1 \\
      684.0 &  8.2 \\
      \bottomrule
    \end{tblr}
  \end{center}
  \vspace{5pt}
  \begin{itemize}
    \item Die Option \mintinline{latex}+vline{2}={2}{-}{text=…}+ ersetzt den Spaltenabstand ab der zweiten Zeile und vor der zweiten Spalte durch \texttt{…}
    \item Die zusätzliche Option \mintinline{latex}+{{-}}+ hinter der Zeilenangabe sorgt dafür, dass dies für alle Zeilen passiert.
    \item Explizite Zeilen-/Spaltenangaben: \mintinline{latex}+vline{2,3}={4,5}{text=…}+ (vierte und fünfte Zeile der Spaltenabstände zwei und drei)
    \item \alert{Darf nur im Header der \texttt{tblr}-Umgebung verwendet werden.}
  \end{itemize}
\end{frame}
