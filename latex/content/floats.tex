\section{Gleitumgebungen}
\begin{frame}[fragile]{
  Gleitumgebungen
  \hfill
  \doc{http://mirrors.ctan.org/macros/latex/contrib/caption/caption-deu.pdf}{caption}
}
  \begin{itemize}
    \item Zum setzen von Elementen, die nicht Fließtext sind
    \item Hauptsächlich Grafiken und Tabellen
    \item Position wird von \LaTeX{} automatisch bestimmt
    \item Nicht auf früherer Seite als umgebender Text
    \item Bekommen meist \mintinline{latex}+\caption+ und \mintinline{latex}+\label+
  \end{itemize}
  \begin{Packages}
    \begin{minted}{latex}
      % Floats innerhalb einer Section halten
      \usepackage[section, below]{placeins}
      \usepackage[…]{caption} % Captions schöner machen
    \end{minted}
  \end{Packages}

  \mintinline{latex}+\FloatBarrier+ kann benutzt werden, um alle vorigen Floats zu setzen.
\end{frame}

\begin{frame}[fragile]{
  Bilder einbinden
  \hfill
  \doc{http://mirrors.ctan.org/macros/latex/required/graphics/grfguide.pdf}{graphicx}
}
  \begin{Packages}
    \begin{minted}{latex}
      \usepackage{graphicx}
    \end{minted}
  \end{Packages}
  \begin{CodeExample}{0.7}
    \begin{minted}{latex}
      \begin{figure}
        \centering
        \includegraphics[width=\textwidth]{logos/pep.pdf}
        \caption{Das Pep-Logo.}
        \label{fig:peplogo}
      \end{figure}
    \end{minted}
  \CodeResult
    \begin{figure}
      \centering
      \includegraphics[width=\textwidth]{logos/pep.pdf}
      \caption{Das PeP-Logo.}
      \label{fig:peplogo}
    \end{figure}
  \end{CodeExample}
  \vspace{5pt}
  \begin{itemize}
    \item Auch möglich: \mintinline{latex}+height=...+, \mintinline{latex}+scale=...+
    \item \mintinline{latex}+\caption+ endet immer mit einem Punkt.
  \end{itemize}
\end{frame}

\begin{frame}[fragile]{
  Subfigures
  \hfill
  \doc{http://mirror.physik-pool.tu-berlin.de/pub/CTAN/macros/latex/contrib/caption/subcaption.pdf}{subcaption}
}
  \begin{Packages}
    \begin{minted}{latex}
    \usepackage{subcaption}
    \end{minted}
  \end{Packages}
  \begin{EmulateArticle}%
    \begin{figure}%
      \begin{subfigure}{0.48\textwidth}%
        \centering%
        \includegraphics[height=0.75cm]{logos/pep.pdf}%
        \caption{PeP-Logo.}%
        \label{fig:pep2}%
      \end{subfigure}%
      \hfill%
      \begin{subfigure}{0.48\textwidth}%
        \centering%
        \includegraphics[height=0.75cm]{logos/tu.pdf}%
        \caption{Das TU-Logo.}%
        \label{fig:TU}%
      \end{subfigure}%
      \caption{Zwei Logos, Abbildung \subref{fig:TU}: das TU-Logo.}\label{fig:logos}%
    \end{figure}%
  \end{EmulateArticle}
\end{frame}

\begin{frame}[fragile]{Subfigures: Code}
  In \LaTeX{} wirkt ein Zeilenende wie ein Leerzeichen,
  dies ist oft unerwünscht und kann durch ein \% am Ende der Zeile behoben werden.

  \begin{block}{Code}
    \begin{minted}{latex}
      \begin{figure}%
        \begin{subfigure}{0.48\textwidth}%
          \centering%
          \includegraphics[height=0.75cm]{logos/pep.pdf}%
          \caption{PeP-Logo.}%
          \label{fig:pep2}%
        \end{subfigure}%
        \hfill% Fills available space in the center -> space between figures
        \begin{subfigure}{0.48\textwidth}%
          \centering%
          \includegraphics[height=0.75cm]{logos/tu.pdf}%
          \caption{Das TU-Logo.}%
          \label{fig:TU}%
        \end{subfigure}%
        \caption{Zwei Logos, Abbildung \subref{fig:TU}: Das TU-Logo.}%
        \label{fig:logos}%
      \end{figure}%
    \end{minted}
  \end{block}
\end{frame}

\begin{frame}[fragile]{Referenzen}
  \begin{block}{Code}
    \begin{minted}{latex}
      \section{Messung mit Apparatur 2}
      \label{sec:apparatur2}
      % .
      \section{Auswertung}
      Wie in \ref{sec:apparatur2} beschrieben, ...
    \end{minted}
  \end{block}
  \begin{itemize}
    \item Auch für Gleichungen, Grafiken, Tabellen
    \item Für Übersichtlichkeit sollten Labels den Typ der Referenz nennen:
      \begin{description}
        \item[Sections]    \texttt{sec:}
        \item[Gleichungen] \texttt{eqn:}
        \item[Abbildungen] \texttt{fig:}
        \item[Tabellen]    \texttt{tab:}
      \end{description}
    \item Bei Gleichungen: \mintinline{latex}+\eqref+ statt \mintinline{latex}+\ref+ → setzt Klammern: \eqref{eqn:maxwell1}
    \item \mintinline{latex}+\label+ immer nach dem, worauf verwiesen wird
  \end{itemize}
\end{frame}

\begin{frame}[fragile]{\texttt{\backslash ref} vs. \texttt{\backslash subref}}
  \begin{CodeExample}{0.49}
    \begin{minted}{latex}
      In Abbildung \ref{fig:logos} sehen Sie zwei Logos.
      In Abbildung \ref{fig:pep2} sehen Sie das PeP-Logo.
      In Abbildung \subref{fig:pep2} sehen Sie das PeP-Logo.
      In \autoref{fig:pep2} sehen Sie das PeP-Logo.
    \end{minted}
    \CodeResult
      \strut
      In Abbildung~\ref{fig:logos} sehen Sie zwei Logos. \\[\baselineskip]
      In Abbildung~\ref{fig:pep2} sehen Sie das PeP-Logo. \\[\baselineskip]
      In Abbildung~\subref{fig:pep2} sehen Sie das PeP-Logo.\\[\baselineskip]
      In Abbildung~\ref{fig:pep2} sehen Sie das PeP-Logo.
  \end{CodeExample}
  \vspace{2em}
  \mintinline{latex}+\subref+ nur in \mintinline{latex}+\caption{…}+ zu Subfigures sinnvoll. \\
  \mintinline{latex}+\autoref+ erfordert eine Sprachoption für \mintinline{latex}+hyperref+: \mintinline{latex}+\usepackage[german, …]{hyperref}+  \\
\end{frame}

\begin{frame}[fragile]{Positionen der Gleitumgebungen}
  \begin{itemize}
    \item \LaTeX{} hat 4 Regionen, in die es Float-Umgebungen platziert
      \begin{description}
        \item[\texttt{h}] here, zwischen Text
        \item[\texttt{t}] top, oben auf einer Seite
        \item[\texttt{b}] bottom, unten auf einer Seite
        \item[\texttt{p}] page, eigene Seite nur für Floats
      \end{description}
    \item Standardmäßig nur \texttt{t,b,p} genutzt
    \item \alert{Nicht} empfohlen: Änderung mit optionalem Argument an Umgebung
    \item Änderung des Standards mit dem Paket \texttt{float}
  \end{itemize}

  \begin{Packages}
    \begin{minted}{latex}
      \usepackage{scrhack} % nach \documentclass

      \usepackage{float}
      \floatplacement{figure}{htbp}
      \floatplacement{table}{htbp}
    \end{minted}
  \end{Packages}
\end{frame}
