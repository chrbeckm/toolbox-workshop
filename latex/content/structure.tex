\section{Struktur}

\begin{frame}[fragile]{Titelseite und Metadaten}
  \LaTeX\ erstellt automatisch eine Titelei aus den Metadaten. \\
  Mit der Klassenoption \mintinline{latex}{titlepage=firstiscover} wird diese  als eigene Seite gesetzt.

  \begin{block}{Neue Klassenoption}
    \begin{minted}{latex}
      \documentclass[…, titlepage=firstiscover, …]{scrartcl}
    \end{minted}
  \end{block}

  \begin{block}{Empfehlung fürs Praktikum:}
    \begin{minted}{latex}
      \title{101 - Das Trägheitsmoment}
      % Mehrere Autoren mit \and:
      \author{Christian Beckmann \and Joshua Luckey}
      \date{Durchführung: 26.09.2017, Abgabe: 29.09.2017}
    \end{minted}
  \end{block}

  \begin{block}{Titelseite generieren}
    \begin{minted}{latex}
      \maketitle
    \end{minted}
  \end{block}
\end{frame}

\begin{frame}[fragile]{Gliederung}
  \LaTeX\ bietet Befehle zum Erstellen von Gliederungsebenen.
  Diese werden automatisch nummeriert und in entsprechend größerer und fetter Schrift gesetzt.

  \begin{block}{Gliederungsebenen für \texttt{scrartcl}}
    \begin{minted}{latex}
      \section{Überschrift}
      \subsection{Überschrift}
      \subsubsection{Überschrift}
      \paragraph{Überschrift}    % wird nicht nummeriert
      \subparagraph{Überschrift} % wird nicht nummeriert
    \end{minted}
  \end{block}
  \begin{block}{Höhere Gliederungsebenen für \texttt{scrreprt} und \texttt{scrbook}}
    \begin{minted}{latex}
      \part{Überschrift}
      \chapter{Überschrift}
      \section{Überschrift}
    \end{minted}
  \end{block}
\end{frame}

\begin{frame}[fragile]{Inhaltsverzeichnis}
  Aus den Gliederungselementen kann automatisch das Inhaltsverzeichnis erzeugt werden.
  \begin{block}{Inhaltsverzeichnis generieren}
    \begin{minted}{latex}
      \tableofcontents
      \newpage
    \end{minted}
  \end{block}
\end{frame}
