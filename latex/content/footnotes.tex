\section{Fußnoten}

\begin{frame}[fragile]{Fußnoten}
  \begin{CodeExample}{0.5}
    \begin{minted}{latex}
      In diesem Versuch werden
      PMTs\footnote{Photo-Multiplier-Tubes}
      eingesetzt.
    \end{minted}
    \CodeResult
      In diesem Versuch werden
      PMTs\footnote{\rmfamily Photo-Multiplier-Tubes}
      eingesetzt.
    \vspace{4cm}
  \end{CodeExample}
  \begin{itemize}
    \item Anpassung von Fußnoten mit dem Paket \texttt{footmisc}
  \end{itemize}
\end{frame}

\begin{frame}[fragile]{Fußnoten in Floats}
  \begin{alertblock}{Vorsicht bei Float-Umgebungen!}
    \begin{minted}{latex}
      \begin{figure}
        \includegraphics[height=0.5cm]{pep.pdf}
        \caption{Bla\protect\footnotemark}
      \end{figure}
      \footnotetext{Unsinnige Caption.}
    \end{minted}
  \end{alertblock}
  \vspace{-1pt}
  \begin{itemize}
    \item \mintinline{latex}+\footnotemark+ an der Stelle wo die Fußnote sein soll
    \item In einer \mintinline{latex}+\caption+ muss dem \mintinline{latex}+\footnotemark+ ein \mintinline{latex}+\protect+ vorangestellt werden.
      \begin{itemize}
        \item Aufpassen, wenn man eine \mintinline{latex}+\listoffigures+ hat
      \end{itemize}
    \item \mintinline{latex}+\footnotetext{...}+ außerhalb der Umgebung für den Text der Fußnote
    \item Fußnoten in Abbildungen sollten vermieden werden.
  \end{itemize}
\end{frame}
