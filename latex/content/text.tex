\section{Text erstellen}

\begin{frame}[fragile]{Text schreiben}
  \begin{block}{Beispiel}
    \begin{minted}{latex}
      % Präambel
      \begin{document}
        Hallo, Welt!

        Dies ist ein dummer Beispieltext.
        Er soll zeigen, dass \LaTeX{} sich nicht um
        Zeilenumbrüche im Code    oder    zuviele
        Leerzeichen kümmert.

        Ein Absatz wird mit einer leeren Code-Zeile
        markiert.
      \end{document}
    \end{minted}
  \end{block}
\end{frame}

\begin{frame}[fragile]{Konventionen für Text}
  \begin{itemize}
    \item Höchstens ein Satz pro Code-Zeile
    \item Absätze werden durch eine Leerzeile markiert
    \item Im Fließtext sollten keine Umbrüche mit \mintinline{latex}+\\+ erzwungen werden
  \end{itemize}
  \begin{alertblock}{Sonderzeichen}
    Viele Sonderzeichen sind \LaTeX-Steuerzeichen.
    Damit diese im Text genutzt werden können, muss meist ein \verb+\+ vorangestellt oder ein Befehl genutzt werden.
  \end{alertblock}
  \begin{CodeExample}{0.7}
    \begin{minted}{latex}
      \# \$ \% \& \_ \{ \}
      \textbackslash \textasciicircum \textasciitilde
    \end{minted}
  \CodeResult
    \strut
    \# \$ \% \& \_ \{ \} \\
    \textbackslash\ \textasciicircum\ \textasciitilde
  \end{CodeExample}
\end{frame}

\begin{frame}[fragile]{Textauszeichnung}
  Änderungen der Schrifteigenschaften sind mit diesen Befehlen möglich:
  \begin{CodeExample}{0.60}
    \begin{minted}{latex}
      \textit{kursiv} \emph{kursiv}
      \textbf{fett}
      \textbf{\textit{fett-kursiv}}
      \textrm{Serifen-Schrift}
      \texttt{Mono-Schrift}
      \textsf{Sans-Serif-Schrift}
      \textsc{Kapitälchen}
    \end{minted}
  \CodeResult
    \strut
    \textit{kursiv} \emph{kursiv} \\
    \textbf{fett} \\
    \textbf{\textit{fett-kursiv}} \\
    \textrm{Serifen-Schrift} \\
    \texttt{Mono-Schrift} \\
    \textsf{Sans-Serif-Schrift} \\
    \textsc{Kapitälchen}
  \end{CodeExample}

  \vspace{1em}
  Diese Befehle sollten sehr selten benutzt werden, semantischer Markup ist besser.
\end{frame}

\begin{frame}[fragile]{Schriftgrößen}
  Gelten immer für den aktuellen Block, z.\,B. in einer Umgebung oder zwischen \mintinline{latex}+{ }+
  \begin{CodeExample}{0.48}
    \begin{minted}{latex}
      {\tiny tiny}
      {\small small}
      {\normalsize normal}
      {\large large}
      {\huge huge}
    \end{minted}
  \CodeResult
    \begin{minipage}[c][5\baselineskip][c]{\textwidth}
      {\tiny tiny}
      {\small small}
      {\normalsize normal}
      {\large large}
      {\huge huge}
    \end{minipage}
  \end{CodeExample}
  \vspace{1em}
  \begin{block}{Alle Größen}
    \begin{minted}{latex}
      \tiny, \scriptsize, \footnotesize, \small, \normalsize, \large, \Large, \LARGE, \huge, \Huge
    \end{minted}
  \end{block}
  Auch diese Befehle sollten nur über semantischen Markup benutzt werden.
\end{frame}

\begin{frame}[fragile]{Inhalt auslagern}
  \begin{block}{Code}
    \begin{minted}{latex}
      \PassOptionsToPackage{
  german,
  unicode,
  pdfusetitle,
  colorlinks,
  linkcolor=vertexDarkRed,
  urlcolor=vertexDarkRed,
  citecolor=vertexDarkRed,
}{hyperref}
\PassOptionsToPackage{
  aux,
}{rerunfilecheck}

\documentclass[
  9pt,
  aspectratio=1610,
]{beamer}
\usetheme{vertex}

\usefonttheme{professionalfonts}
\usepackage[english, french, ngerman]{babel}

\usepackage{fontspec}
\setmainfont{Latin Modern Roman}[
  SmallCapsFont = {Latin Modern Roman Caps},
]
\newfontfamily{\latinmodernroman}{Latin Modern Roman}[
  SmallCapsFont = {Latin Modern Roman Caps},
]
\DeclareTextFontCommand{\latinmodernrm}{\latinmodernroman}
\newfontfamily{\latinmodernsans}{Latin Modern Sans}
\DeclareTextFontCommand{\latinmodernsf}{\latinmodernsans}
\newfontfamily{\latinmodernmono}{Latin Modern Mono}
\DeclareTextFontCommand{\latinmoderntt}{\latinmodernmono}

\usepackage{fontawesome5}

\usepackage{amsmath}
\usepackage{mathtools}
\usepackage{amssymb}
\usepackage{mleftright}

\usepackage[
  math-style=ISO,
  bold-style=ISO,
  sans-style=italic,
  nabla=upright,
  partial=upright,
  mathrm=sym,
]{unicode-math}
\setmathfont{Latin Modern Math}[version=normal]
\setmathfont{XITS Math}[version=xits]
\setmathfont{XITS Math}[version=xitsss1, StylisticSet=1]
\mathversion{normal}

\usepackage[
  locale=DE,
  separate-uncertainty=true,
  per-mode=symbol-or-fraction,
]{siunitx}

\usepackage[
  version=4,
  math-greek=default,
  text-greek=default,
]{mhchem}

\usepackage[
  german=quotes,
  autostyle,
]{csquotes}
\usepackage{xfrac}

\usepackage{tabularray}
\UseTblrLibrary{booktabs, siunitx, varwidth}
\usepackage{threeparttable}

\usepackage{graphicx}

\usepackage[
  compatibility=false,
]{caption}
\usepackage{subcaption}

\usepackage{xcolor}
\usepackage{metalogo}
\usepackage{pdflscape}

\usepackage{fancyvrb}
\usepackage[highlightmode=immediate]{minted}
\setminted{
  autogobble,
  breaklines,
  stripnl=true,
}
\usemintedstyle{toolbox}

\usepackage[
  theorems,
  many,
]{tcolorbox}

\usepackage{tikz}
\usetikzlibrary{
  arrows,
  arrows.meta,
  graphs,
  graphdrawing,
  positioning,
  shadows,
  shapes,
}

\usegdlibrary{trees}
\usepackage[compat=1.1.0]{tikz-feynman}

\usepackage[
  shortcuts,
]{extdash}

\usepackage[noframe]{showframe}
\usepackage{bookmark}

\tcbset{
  noparskip,
  colback=white,
  colframe=vertexDarkGrey,
  boxsep=1mm,
  coltext=black,
  coltitle=white,
  left=0mm,
  right=0mm,
  top=0.5mm,
  toptitle=-1mm,
  bottom=0mm,
  bottomtitle=-1mm,
  sharp corners,
  before upper={\raggedright},
  before title={\raggedright},
}

\tikzset{
  invisible/.style={opacity=0,text opacity=0},
  visible on/.style={alt={#1{}{invisible}}},
  alt/.code args={<#1>#2#3}{%
    \alt<#1>{\pgfkeysalso{#2}}{\pgfkeysalso{#3}} % \pgfkeysalso doesn't change the path
  },
}
\setbeamertemplate{bibliography item}{\insertbiblabel}

\ExplSyntaxOn

\RenewDocumentEnvironment {block} {m o} {
  \IfValueTF{#2}{
    \begin{tcolorbox}[
      adjusted~title=#1,
      #2,
    ]
  }{
    \begin{tcolorbox}[
      adjusted~title=#1,
    ]
  }
}{
  \end{tcolorbox}
}

\RenewDocumentEnvironment {alertblock} {m o} {
  \IfValueTF{#2}{
    \begin{tcolorbox}[
      adjusted~title=#1,
      colframe=vertexDarkRed,
      #2,
    ]
  }{
    \begin{tcolorbox}[
      adjusted~title=#1,
      colframe=vertexDarkRed,
    ]
  }
}{
  \end{tcolorbox}
}

\RenewDocumentEnvironment {exampleblock} {m o} {
  \IfValueTF{#2}{
    \begin{tcolorbox}[
      adjusted~title=#1,
      colframe=blue!80!black,
      #2,
    ]
  }{
    \begin{tcolorbox}[
      adjusted~title=#1,
      colframe=blue!80!black,
    ]
  }
}{
  \end{tcolorbox}
}

\newcounter{heightgroup}


% see https://tex.stackexchange.com/a/374564/59716
\NewDocumentEnvironment {BeamerCodeFrame} {O{} g} {
  \IfValueTF{#2}{
    \begin{frame}[fragile, environment=BeamerCodeFrame, #1]{#2}
  }{
    \begin{frame}[fragile, environment=BeamerCodeFrame, #1]
  }
}{
  \end{frame}
}


\NewDocumentEnvironment {CodeExplanation} {o m O{Code} O{Erklärung} D<>{}} {
  \NewDocumentCommand \Explanation {D<>{}} {
      \end{tcolorbox}
    \end{column}
    \begin{column}{
      \IfValueTF {#1} {
        #1
      }{
        \fp_eval:n {1 - 0.02 - #2}
      }
      \textwidth
    }
    \begin{tcolorbox}[
      equal~height~group=\theheightgroup,
      adjusted~title=#4,
      valign=top,
      ##1,
    ]
  }
  \begin{columns}[onlytextwidth, t]
    \begin{column}{#2 \textwidth}
      \begin{tcolorbox}[
        adjusted~title=#3,
        equal~height~group=\theheightgroup,
        valign=top,
        #5,
      ]
}{
      \end{tcolorbox}
    \end{column}
  \end{columns}
  \stepcounter{heightgroup}
}


\NewDocumentEnvironment {CodeExample} {o m O{Code} O{Ergebnis} D<>{}} {
  \NewDocumentCommand \CodeResult {D<>{}} {
      \end{tcolorbox}
    \end{column}
    \begin{column}{
      \IfValueTF {#1} {
        #1
      }{
        \fp_eval:n {1 - 0.02 - #2}
      }
      \textwidth
    }
      \begin{tcolorbox}[
        adjusted~title=#4,
        colframe=gray!40,
        coltitle=black,
        equal~height~group=\theheightgroup,
        valign=top,
        ##1,
      ]
        \begin{EmulateArticle}
  }
  \begin{columns}[onlytextwidth, t]
    \begin{column}{#2 \textwidth}
      \begin{tcolorbox}[
        adjusted~title=#3,
        equal~height~group=\theheightgroup,
        valign=top,
        #5,
      ]
}{
        \end{EmulateArticle}
      \end{tcolorbox}
    \end{column}
  \end{columns}
  \stepcounter{heightgroup}
}

\NewDocumentEnvironment {Packages} {} {
  \begin{block}{Benötigte~Pakete}
}{
  \end{block}
}

\makeatletter
\NewDocumentEnvironment {EmulateArticle} {} {
  \rmfamily
  \newfontfamily\frenchfontsf{Latin~Modern~Roman}
  \newfontfamily\englishfontsf{Latin~Modern~Roman}
  \let\textrm=\latinmodernrm
  \let\textsf=\latinmodernsf
  \let\texttt=\latinmoderntt
  \renewcommand\thempfootnote{\arabic{mpfootnote}}

  \color{black}
  \setbeamercolor{item}{fg=black}
  \setbeamercolor{itemize/enumerate~body}{fg=black}
  \setbeamercolor{itemize/enumerate~subbody}{fg=black}
  \setbeamercolor{itemize/enumerate~subsubbody}{fg=black}
  \setbeamercolor{description~item}{fg=black}
  \setbeamercolor{enumerate~item}{fg=black}
  \setbeamercolor{itemize~item}{fg=black}
  \setbeamercolor{normal~text}{fg=black}
  \setbeamercolor{block~body}{fg=black}
  \setbeamerfont{item}{family=\rmfamily, size=\normalsize}
  \setbeamerfont{itemize/enumerate~body}{family=\rmfamily, size=\normalsize}
  \setbeamerfont{itemize/enumerate~subbody}{family=\rmfamily, size=\normalsize}
  \setbeamerfont{itemize/enumerate~subsubbody}{family=\rmfamily, size=\normalsize}
  \setbeamerfont{description~item}{series=\bfseries}
  \setbeamertemplate{itemize~item}{\bullet}
  \setbeamertemplate{itemize~subitem}{--}
  \setbeamertemplate{itemize~subsubitem}{\textasteriskcentered}
  \setbeamertemplate{enumerate~item}{\theenumi.}
  \setbeamertemplate{enumerate~subitem}{\alph{enumii})}
  \setbeamertemplate{enumerate~subsubitem}{\roman{enumiii}.}

  \setbeamerfont{footnote}{family=\rmfamily}
  \setbeamerfont{footnote~mark}{family=\rmfamily}

  \setbeamertemplate{caption}[numbered]
  \setbeamertemplate{caption~label~separator}[colon]
  \setbeamercolor{caption}{fg=black}
  \setbeamerfont{caption}{family=\rmfamily}
  \setbeamercolor{caption~name}{fg=black}
  \setbeamerfont{caption~name}{family=\rmfamily, series=\bfseries}

  % from amsmath.dtx
  % \def\maketag@@@#1{\hbox{\m@th\normalfont#1}}
  \def\maketag@@@##1{\hbox{\m@th\normalfont\rmfamily##1}}
  \hypersetup{linkcolor=black}
}{
}
\makeatother

\NewDocumentEnvironment {CenterStrip} {O{\textwidth} m}
{
  \begin{minipage}[c][#2\baselineskip][c]{#1}
}{
  \end{minipage}
}

\NewDocumentCommand \OverfullCenter {+m}
{
	\noindent
	\makebox[\linewidth]{#1}
}

\tikzstyle{buttonstyle} = [
  align=center,
  rectangle,
  fill=black!10,
  draw=black!80,
  drop~shadow,
  font={\bfseries},
]

\NewDocumentCommand \button {m}
{
  \tikz[
    baseline={([yshift=-.8ex]current~bounding~box.center)},
  ]{
    \node[buttonstyle] {\normalsize #1};
  }
}

\NewDocumentCommand \doc {m m}
{
  \button{\href{#1}{Doku:~\texttt{#2}}}
}

% used to show headline at beginning of each logical section
\NewDocumentCommand \headlineframe {m}
{
  \begin{frame}
    \begin{center}
      \Huge\color{vertexDarkRed}#1
    \end{center}
  \end{frame}
}

\NewDocumentCommand \removedisplayskip {}
{
  \setlength\abovedisplayskip{0ex}
  \setlength\belowdisplayskip{0ex}
  \setlength\abovedisplayshortskip{0ex}
  \setlength\belowdisplayshortskip{0ex}
}

\makeatletter
\seteverylogo
{
  \setlogodrop{0.7ex}
  \setLaTeXa{{\scshape a}}
  \fp_compare:nNnTF {\f@size} > {10}
  {
    \setlogokern{La}{-0.33em}
  }
  {
    \fp_compare:nNnTF {\f@size} < {10}
    {
      \setlogokern{La}{-0.4em}
    }
    {
      \setlogokern{La}{-0.36em}
    }
  }
}
\makeatother

\let\TeXold=\TeX
\def\TeX{\latinmodernrm{\TeXold}}
\def\eTeX{$ε$-\TeX}
\let\LaTeXold=\LaTeX
\def\LaTeX{\latinmodernrm{\LaTeXold}}
\let\XeTeXold=\XeTeX
\def\XeTeX{\latinmodernrm{\XeTeXold}}
\let\XeLaTeXold=\XeLaTeX
\def\XeLaTeX{\latinmodernrm{\XeLaTeXold}}
\let\LuaTeXold=\LuaTeX
\def\LuaTeX{\latinmodernrm{\LuaTeXold}}
\let\LuaLaTeXold=\LuaLaTeX
\def\LuaLaTeX{\latinmodernrm{\LuaLaTeXold}}
\def\pdfTeX{\latinmodernrm{pdf\TeX}}
\def\BibTeX{\latinmodernrm{\textsc{Bib}\TeX}}
\def\BibLaTeX{\latinmodernrm{Bib\LaTeX}}

% macro examples
\NewDocumentCommand \dif {m}
{
  \mathinner{\symup{d} #1}
}
\NewDocumentCommand \Dif {m}
{
  \mathinner{\symup{D} #1}
}
\NewDocumentCommand \del {m}
{
  \mathinner{\symup{\delta} #1}
}
\NewDocumentCommand \Del {m}
{
  \mathinner{\symup{\Delta} #1}
}

\let\vaccent=\v
\RenewDocumentCommand \v {}
{
  \TextOrMath{
    \vaccent
  }{
    \symbf
  }
}

\let\ltext=\l
\RenewDocumentCommand \l {}
{
  \TextOrMath { \ltext }{ \mleft }
  % \mathopen{}\mathclose\bgroup\left
}
\let\raccent=\r
\RenewDocumentCommand \r {}
{
  \TextOrMath { \raccent }{ \mright }
  % \aftergroup\egroup\right
}

\AtBeginDocument{
  \let\symRe=\Re
  \RenewDocumentCommand \Re {}
  {
    \operatorname{Re}
  }
  \let\symIm=\Im
  \RenewDocumentCommand \Im {}
  {
    \operatorname{Im}
  }
}

\DeclarePairedDelimiter{\abs}{\lvert}{\rvert}
\DeclarePairedDelimiter{\norm}{\lVert}{\rVert}
\DeclarePairedDelimiter{\bra}{\langle}{\rvert}
\DeclarePairedDelimiter{\ket}{\lvert}{\rangle}
\DeclarePairedDelimiterX{\braket}[2]{\langle}{\rangle}{#1\delimsize|#2}

\ExplSyntaxOff


\author{PeP et al.\ Toolbox Workshop}
\date{2023}
\institute[Pep et al.\ e.V.]{\includegraphics[width=0.4\textwidth]{logos/pep.pdf}}

      \begin{document}
        \input{Teil1.tex}
        \input{Teil2.tex}
        % .
      \end{document}
    \end{minted}
  \end{block}
  \begin{itemize}
    \item Verschachtelung möglich
    \item Zur Aufteilung größerer Dokumente (z.B. diese Präsentation)
    \item Für häufig wiederverwendeten Code (Header, Erläuterungen zu Fehlerrechnung, …)
    \item Für per Skript erzeugte Tabelleninhalte
  \end{itemize}
\end{frame}

\begin{frame}[fragile]{
  Anführungszeichen
  \hfill
  \doc{http://mirrors.ctan.org/macros/latex/contrib/csquotes/csquotes.pdf}{csquotes}
}
  Die richtigen Anführungszeichen, wo die Satzzeichen hingehören und vieles mehr hängt von der Sprache ab.
  So macht man es richtig:
  \begin{Packages}
    \begin{minted}{latex}
      % babel mit anderen Sprachen laden
      \usepackage[english, french, ngerman]{babel}
      \usepackage[autostyle]{csquotes}    % babel
    \end{minted}
  \end{Packages}
  \begin{CodeExample}{0.60}
    \begin{minted}{latex}
      foo \enquote{bar} baz
      \enquote{foo \enquote{bar} baz}
      \foreignlanguage{english}{\enquote{foo}}
      \foreignlanguage{french}{\enquote{foo}}
      % siehe Kapitel über Bibliographie
      \textcquote{numpy}{foo}
    \end{minted}
  \CodeResult
    \strut
    foo \enquote{bar} baz \\
    \enquote{foo \enquote{bar} baz} \\
    \foreignlanguage{english}{\enquote{foo}}\\
    \foreignlanguage{french}{\enquote{foo}}\\[\baselineskip]
    % siehe Kapitel über Bibliographie
    \textcquote{numpy}{foo}
  \end{CodeExample}
\end{frame}
