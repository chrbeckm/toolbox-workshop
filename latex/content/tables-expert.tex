\subsection{Tabellen: Expert}

\begin{frame}[fragile]{
  Lange Tabellen
  \hfill
  \doc{http://mirrors.ctan.org/macros/latex/contrib/tabularray/tabularray.pdf}{tabularray}
}
   Manchmal sind Tabellen zu lang für eine Seite. Hierbei sorgt die \texttt{longtblr}-Umgebung für Abhilfe:

   \begin{block}{Code}
      \begin{minted}{latex}
        \begin{longtblr}[
           caption = {Eine lange Tabelle.},
           label = {tab:long_table},
         ]{
           colspec = {S[table-format=3.1] S[table-format=1.3] S[table-format=1.3]}, 
           rowhead = 1, row{1} = {guard, mode=math},
         }
         \toprule
         x \mathbin{/} \unit{\milli\meter} & I_1 \mathbin{/} \unit{\micro\ampere} & I_2 \mathbin{/} \unit{\micro\ampere} \\
         \midrule
         -10.0  &  0.009  &  3.600 \\
          -9.5  &  0.009  &  3.200 \\
           ...       ...       ...
          -2.5  &  0.175  &  0.032 \\
          -2.2  &  0.215  &  0.027 \\
         \bottomrule
        \end{longtblr}
      \end{minted}
   \end{block}
\end{frame}

\frametitle{Ergebnis}
\begin{longtblr}[
  caption = {Eine lange Tabelle.},
  label = {tab:long_table},
]{
  colspec = {S[table-format=3.1] S[table-format=1.3] S[table-format=1.3]}, 
  rowhead = 1, row{1} = {guard, mode=math},
}
  \toprule
  x \mathbin{/} \unit{\milli\meter} & I_1 \mathbin{/} \unit{\micro\ampere} & I_2 \mathbin{/} \unit{\micro\ampere} \\
  \midrule
  -10.0  &  0.009  &  3.600 \\
   -9.5  &  0.009  &  3.200 \\
   -9.0  &  0.011  &  2.550 \\
   -8.5  &  0.013  &  1.500 \\
   -8.0  &  0.016  &  0.720 \\
   -7.5  &  0.020  &  0.250 \\
   -7.0  &  0.020  &  0.088 \\
   -6.5  &  0.021  &  0.105 \\
   -6.0  &  0.031  &  0.135 \\
   -5.5  &  0.032  &  0.125 \\
   -5.0  &  0.025  &  0.072 \\
   -4.5  &  0.042  &  0.038 \\
   -4.0  &  0.072  &  0.038 \\
   -3.7  &  0.073  &  0.056 \\
   -3.4  &  0.066  &  0.062 \\
   -3.1  &  0.068  &  0.032 \\
   -2.8  &  0.105  &  0.028 \\
   -2.5  &  0.175  &  0.032 \\
   -2.2  &  0.215  &  0.027 \\
   \bottomrule
\end{longtblr}
\begin{itemize}
  \item Die Tabelle wird automatisch auf der nächsten Seite fortgeführt.
  \item \alert{Wichtig: \texttt{longtblr} ohne die \texttt{table-Umgebung nutzen}.}
  \begin{itemize}
    \item \alert{Setzt \texttt{caption} und \texttt{label} in den Tabellenoptionen} \mintinline{latex}+\begin{longtblr}[...]+.
  \end{itemize}
\end{itemize}
\pagebreak

\begin{frame}[fragile]{Anmerkungen}
  \begin{alertblock}{Vorsicht!}
     \begin{itemize}
       \item Zur Zeit ist \texttt{tabularray} nicht komplett im Sprachpaket \texttt{babel} übersetzt
     \end{itemize}
  \end{alertblock}
  \begin{itemize}
    \item Setzt folgende Optionen in der Präambel des Dokuments um \texttt{longtblr}-Umgebungen auf Deutsch zu erhalten
  \end{itemize}
  \begin{block}{Neue Optionen}
    \begin{minted}{latex}
       \DefTblrTemplate{contfoot-text}{normal}{Weiter auf der nächsten Seite}
       \SetTblrTemplate{contfoot-text}{normal}
       \DefTblrTemplate{conthead-text}{normal}{(Fortsetzung)}
       \SetTblrTemplate{conthead-text}{normal}
     \end{minted}
  \end{block}
\end{frame}
